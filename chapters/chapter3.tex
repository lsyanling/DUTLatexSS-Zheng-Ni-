\BiChapter{图表、公式、代码及参考文献的编制方法}{The Compile Method of Chart, Formula , Code and Reference }
\BiSection{向文档中插入图像}{Inserts an Image Into the Document}
\BiSubsection{单张图片的插入}{Insertion of a Single Image}
在学位论文中,可插入PDF、EPS、PNG、JPG格式的图片。以图3.1的插入为例,插入代码为3.1所示,其中\textbackslash centering表示图片居中,\textbackslash includegraphics[...]\{...\}导入图片并定制图片大小,\textbackslash\{...\}指定图片标题,而\textbackslash lable\{...\}为图片加上引用标签:
\begin{lstlisting}[caption={示例插图代码}]
\begin{figure}[!ht]
\centering
\includegraphics{figures/figure4}
\bicaption{图片}{Picture}  \label{fig4:diagram}
\end{figure}
\end{lstlisting}

\begin{figure}[!ht]
	\centering
	\includegraphics{figures/figure4}
	\bicaption{图片}{Picture}  \label{fig4:diagram}
\end{figure}
\BiSubsection{两张并列图片的插入}{The Insertion of Two Side-by-side Pictures}
插入两幅并列子图的例子如图3.2所示。这两个水平并列放置的图共享一个“图标题”,且有各自的小标题,并列子图的功能是使用subfigure宏包提供的。
\begin{lstlisting}[caption={插入并列子图代码}]
\begin{figure}[!ht]
\centering
\subfigure[热流耦合数值模拟]{\includegraphics[width=0.45\textwidth]{figures/figure8}}
\subfigure[热固耦合数值模拟]{\includegraphics[width=0.45\textwidth]{figures/figure9}}
\bicaption{数值模拟图像}{Numerical simulation image}
\end{figure}
\end{lstlisting}
\begin{figure}[!ht]
	\centering
	\subfigure[热流耦合数值模拟]{\includegraphics[width=0.45\textwidth]{figures/figure8}}
	\subfigure[热固耦合数值模拟]{\includegraphics[width=0.45\textwidth]{figures/figure9}}
	\bicaption{数值模拟图像}{Numerical simulation image}
\end{figure}
更多关于Latex插图的例子可以参考《\LaTeX 插图指南》。
\BiSection{表格的例子}{Example of Table Compile}
表在正文中的常用格式如表\ref{tab3:category},使用三线表。\par
\begin{table}[!ht]
	\small
	\centering
	\bicaption{国内外各返回式航天器热控设计情况}{ Design of thermal control systems of spacecraft for different countries} \label{tab3:category}
	\begin{tabular}{m{5em}<{\centering}m{5em}<{\centering}m{5em}<{\centering}m{5em}<{\centering}m{5em}<{\centering}m{5em}<{\centering}}
		\toprule[2pt]
		项目、指标 &月地高速再入返回器&传统返回式卫星回收舱&神舟飞船&国外载人飞船&航天飞机 \\
		\midrule[1pt]
		回收舱气密性&半密封舱&非密封舱&密封舱&密封舱&密封舱\\
		回收舱长期热耗(W)&整器150&5 – 25&约1000(含宇航员)&约1000(含宇航员)&1500以上\\
		热控方案&基于柔性自适应“热开关”的新型热控方案&被动热控设计为主、电加热主动热控设计为辅&泵驱单相流体回路+对流通风&泵驱单相流体回路+对流通风&泵驱单相流体回路+对流通风+主动式相变系统\\
		\bottomrule[2pt]
	\end{tabular}
\end{table}
表格的绘制需要知道一些基本命令的用法,比如“\&”具有对齐作用,\textbackslash multicolumn用来合并行,\textbackslash multirow用来合并列,\textbackslash hline表示加入横线进行分隔,还有许多命令这里就不一一展开说明,这里给出一些表格绘制的例子进行示范:
\begin{lstlisting}[caption={表3.2绘制代码}]
\begin{table}[!ht]
\small
\centering
\bicaption{国际单位制的辅助单位况}{ Assistant units of International System of Units} 
\begin{tabular}{m{6em}<{\centering}m{6em}<{\centering}m{6em}<{\centering}}
\toprule[2pt]
量的名称&单位名称&单位符号 \\
\midrule[1pt]
平面角&弧度&rad\\
立体角&球面度&sr\\
\bottomrule[2pt]
\end{tabular}
\end{table}
\end{lstlisting}
\begin{table}[!ht]
	\small
	\centering
	\bicaption{国际单位制的辅助单位况}{ Assistant units of International System of Units} 
	\begin{tabular}{m{6em}<{\centering}m{6em}<{\centering}m{6em}<{\centering}}
		\toprule[2pt]
		量的名称&单位名称&单位符号 \\
		\midrule[1pt]
		平面角&弧度&rad\\
		立体角&球面度&sr\\
		\bottomrule[2pt]
	\end{tabular}
\end{table}
\begin{lstlisting}[caption={表3.3绘制代码}]
\begin{table}[!ht]
\small
\centering
\bicaption{国际单位制中具有专门名称的导出单位}{Export units of special name in International System of Units} 
\begin{tabular}{m{8em}<{\centering}m{6em}<{\centering}m{4em}<{\centering}m{6em}<{\centering}} 
\toprule[2pt]
量的名称&单位名称&单位符号&其他表示式例 \\
\midrule[1pt]
频率&赫[兹]&Hz&$\rm{s^{-1}}$\\
力;重力&牛[顿]&N&$\rm{kg·m/s^2}$\\
压力,压强;应力&	帕[斯卡]&	Pa&$\rm{N/m^2}$\\
能量;功;热	&焦[耳]&	J&	N·m\\
功率;辐射通量&	瓦[特]&	W&	J/s\\
电荷量	&库[仑]&	C&	A·s\\
电位;电压;电动势&	伏[特]&	V&	W/A\\
电容&	法[拉]&	F&	C/V\\
电阻&	欧[姆]&	Ω&	V/A\\
电导&	西[门子]&	S&	A/V\\
磁通量	&韦[伯]&	Wb&	V·s\\
磁通量密度,磁感应强度	&特[斯拉]&	T&$	\rm{Wb/m^2}$\\
电感&	亨[利]&	H&	Wb/A\\
摄氏温度&	摄氏度&	℃	&   \\
光通量&	流明&	lm&	cd·sr\\
光照度	&勒[克斯]&	lx&	$\rm{lm/m^2}$\\
放射性活度&	贝可[勒尔]&	Bq&$\rm{s^{-1}}$\\
吸收剂量&	戈[瑞]&	Gy&	J/kg\\
剂量当量&	希[沃特]&	Sv&	J/kg\\
\bottomrule[2pt]
\end{tabular}
\end{table}
\end{lstlisting}
\begin{table}[!ht]
	\small
	\centering
	\bicaption{国际单位制中具有专门名称的导出单位}{Export units of special name in International System of Units} 
	\begin{tabular}{m{8em}<{\centering}m{6em}<{\centering}m{4em}<{\centering}m{6em}<{\centering}} 
		\toprule[2pt]
		量的名称&单位名称&单位符号&其他表示式例 \\
		\midrule[1pt]
		频率&赫[兹]&Hz&$\rm{s^{-1}}$\\
		力;重力&牛[顿]&N&$\rm{kg·m/s^2}$\\
		压力,压强;应力&	帕[斯卡]&	Pa&$\rm{N/m^2}$\\
		能量;功;热	&焦[耳]&	J&	N·m\\
		功率;辐射通量&	瓦[特]&	W&	J/s\\
		电荷量	&库[仑]&	C&	A·s\\
		电位;电压;电动势&	伏[特]&	V&	W/A\\
		电容&	法[拉]&	F&	C/V\\
		电阻&	欧[姆]&	Ω&	V/A\\
		电导&	西[门子]&	S&	A/V\\
		磁通量	&韦[伯]&	Wb&	V·s\\
		磁通量密度,磁感应强度	&特[斯拉]&	T&$	\rm{Wb/m^2}$\\
		电感&	亨[利]&	H&	Wb/A\\
		摄氏温度&	摄氏度&	℃	&   \\
		光通量&	流明&	lm&	cd·sr\\
		光照度	&勒[克斯]&	lx&	$\rm{lm/m^2}$\\
		放射性活度&	贝可[勒尔]&	Bq&$\rm{s^{-1}}$\\
		吸收剂量&	戈[瑞]&	Gy&	J/kg\\
		剂量当量&	希[沃特]&	Sv&	J/kg\\
		\bottomrule[2pt]
	\end{tabular}
\end{table}
\begin{lstlisting}[caption={表3.4绘制代码}]
\begin{table}[!ht]
\small
\centering
\bicaption{国际单位制的基本单位}{ Basic units of International System of Units} 
\begin{tabular}{m{6em}<{\centering}m{6em}<{\centering}m{6em}<{\centering}}
\toprule[2pt]
量的名称&单位名称&单位符号 \\
\midrule[1pt]
长度	&米&	m\\
质量&	千克(公斤)&	kg\\
时间&	秒&	s\\
电流&	安[培]&	A\\
热力学温度&	开[尔文]&	K\\
物质的量&	摩[尔]&	mol\\
发光强度&	坎[德拉]&	cd\\
\bottomrule[2pt]
\end{tabular}
\end{table}
\end{lstlisting}
\begin{table}[!ht]
	\small
	\centering
	\bicaption{国际单位制的基本单位}{ Basic units of International System of Units} 
	\begin{tabular}{m{6em}<{\centering}m{6em}<{\centering}m{6em}<{\centering}}
		\toprule[2pt]
		量的名称&单位名称&单位符号 \\
		\midrule[1pt]
		长度	&米&	m\\
		质量&	千克(公斤)&	kg\\
		时间&	秒&	s\\
		电流&	安[培]&	A\\
		热力学温度&	开[尔文]&	K\\
		物质的量&	摩[尔]&	mol\\
		发光强度&	坎[德拉]&	cd\\
		\bottomrule[2pt]
	\end{tabular}
\end{table}
\begin{lstlisting}[caption={表3.5绘制代码}]
\begin{table}[!ht]
\small
\centering
\bicaption{国家选定的非国际单位制单位}{Non-International System of Units adopted by the nation} 
\begin{tabular}{m{5em}<{\centering}m{7em}<{\centering}m{5em}<{\centering}m{11em}<{\centering}m{5em}<{\centering}m{5em}<{\centering}}
\toprule[2pt]
量的名称&	单位名称&	单位符号&	换算关系和说明 \\
\midrule[1pt]
\multirow{3}*{时间}&分&min&1 min = 60 s\\
&[小]时&	h&1 h = 60 min= 3600 s\\
&天(日)&d	&	1 d = 24 h= 86400 s\\
\multirow{3}*{平面角}&[角]秒&(")&1" = (π / 648000) rad\\
&[角]分&(')&1' = 60"= (π / 10800) rad\\
&度&(°)	&	1 ° = 60' = (π / 180) rad\\
旋转速度&	转每分&	r/min&	1 r/min = (1 / 60)$\rm{s^{-1}}$\\
\multirow{2}*{长度}&\multirow{2}*{海里}&\multirow{2}*{n mile}&
mile = 1852 m\\&&&(只用于航行)\\
\multirow{3}*{速度}&\multirow{3}*{节}&\multirow{3}*{kn}&
1 kn=1 n mile/h\\&&&= (1852 / 3600) m/s\\&&&(只用于航行)\\
\multirow{2}*{质量}&吨&t&1 t=$10^3$ kg\\
&原子质量单位&u&1 u≈1.6605655 × $10^{-27}\rm{kg}$\\
体积&	升&	L,(1)&	1 L = $10^{-3}\rm{ m^3}$\\
能&	电子伏&	eV&	1 eV≈1.6021892 × $10^{-19}$J\\
级差	&分贝&	dB	&\\
级密度	&特[克斯]&	tex&	1 tex=1 g/km\\	
\bottomrule[2pt]
\end{tabular}
\end{table}
\end{lstlisting}
\begin{table}[!ht]
	\small
	\centering
	\bicaption{国家选定的非国际单位制单位}{Non-International System of Units adopted by the nation} 
	\begin{tabular}{m{5em}<{\centering}m{7em}<{\centering}m{5em}<{\centering}m{11em}<{\centering}m{5em}<{\centering}m{5em}<{\centering}}
		\toprule[2pt]
		量的名称&	单位名称&	单位符号&	换算关系和说明 \\
		\midrule[1pt]
		\multirow{3}*{时间}&分&min&1 min = 60 s\\
		&[小]时&	h&1 h = 60 min= 3600 s\\
		&天(日)&d	&	1 d = 24 h= 86400 s\\
		\multirow{3}*{平面角}&[角]秒&(")&1" = (π / 648000) rad\\
		&[角]分&(')&1' = 60"= (π / 10800) rad\\
		&度&(°)	&	1 ° = 60' = (π / 180) rad\\
		旋转速度&	转每分&	r/min&	1 r/min = (1 / 60)$\rm{s^{-1}}$\\
		\multirow{2}*{长度}&\multirow{2}*{海里}&\multirow{2}*{n mile}&
		mile = 1852 m\\&&&(只用于航行)\\
		\multirow{3}*{速度}&\multirow{3}*{节}&\multirow{3}*{kn}&
		1 kn=1 n mile/h\\&&&= (1852 / 3600) m/s\\&&&(只用于航行)\\
		\multirow{2}*{质量}&吨&t&1 t=$10^3$ kg\\
		&原子质量单位&u&1 u≈1.6605655 × $10^{-27}\rm{kg}$\\
		体积&	升&	L,(1)&	1 L = $10^{-3}\rm{ m^3}$\\
		能&	电子伏&	eV&	1 eV≈1.6021892 × $10^{-19}$J\\
		级差	&分贝&	dB	&\\
		级密度	&特[克斯]&	tex&	1 tex=1 g/km\\	
		\bottomrule[2pt]
	\end{tabular}
\end{table}
\begin{lstlisting}[caption={表3.6绘制代码}]
\begin{table}[!ht]
\small
\centering
\bicaption{用于构成十进倍数和分数单位的词头}{ Used prefixes to make up of denary multiples and subdivisions of the units} 
\begin{tabular}{m{6em}<{\centering}m{6em}<{\centering}m{6em}<{\centering}}
\toprule[2pt]
所表示的因数&	词头名称&	词头符号\\
\midrule[1pt]
$\rm10^{18}$&	艾[克萨]&	E\\
$\rm10^{15}$&	拍[它]&	P\\
$\rm10^{12}$&	太[拉]&	T\\
$\rm10^9$&	吉[咖]&	G\\
$\rm10^6$	&兆&	M\\
$\rm10^3$&	千&	K\\
$\rm10^2$&	百&	h\\
$\rm10^1$	&十&	da\\
$\rm10^{-1}$&	分&	d\\
$\rm10^{-2}$&	厘&	c\\
$\rm10^{-3}$&	毫&	m\\
$\rm10^{-6}$&	微&	μ\\
$\rm10^{-9}$&	纳[诺]&	n\\
$\rm10^{-12}$&	皮[可]&	p\\
$\rm10^{-15}$&	飞[母托]&	f\\
$\rm10^{-18}$&	阿[托]&	a\\	
\bottomrule[2pt]
\end{tabular}
\end{table}
\end{lstlisting}
\begin{table}[!ht]
	\small
	\centering
	\bicaption{用于构成十进倍数和分数单位的词头}{ Used prefixes to make up of denary multiples and subdivisions of the units} 
	\begin{tabular}{m{6em}<{\centering}m{6em}<{\centering}m{6em}<{\centering}}
		\toprule[2pt]
		所表示的因数&	词头名称&	词头符号\\
		\midrule[1pt]
		$\rm10^{18}$&	艾[克萨]&	E\\
		$\rm10^{15}$&	拍[它]&	P\\
		$\rm10^{12}$&	太[拉]&	T\\
		$\rm10^9$&	吉[咖]&	G\\
		$\rm10^6$	&兆&	M\\
		$\rm10^3$&	千&	K\\
		$\rm10^2$&	百&	h\\
		$\rm10^1$	&十&	da\\
		$\rm10^{-1}$&	分&	d\\
		$\rm10^{-2}$&	厘&	c\\
		$\rm10^{-3}$&	毫&	m\\
		$\rm10^{-6}$&	微&	μ\\
		$\rm10^{-9}$&	纳[诺]&	n\\
		$\rm10^{-12}$&	皮[可]&	p\\
		$\rm10^{-15}$&	飞[母托]&	f\\
		$\rm10^{-18}$&	阿[托]&	a\\	
		\bottomrule[2pt]
	\end{tabular}
\end{table}
\BiSection{公式的插入}{The Insertion of Formula}
公式的输入非常简单,只要在以下代码的相应位置改成自己要输入的公式即可。
\begin{lstlisting}[caption={公式插入代码}]
\begin{equation}
St=\frac{fd}{v}=\frac{f\overline{d_{32}}}{\overline{Q^{''}}}
\label{eq:St}
\end{equation}
\end{lstlisting}
\begin{equation}
St=\frac{fd}{v}=\frac{f\overline{d_{32}}}{\overline{Q^{''}}}
\label{eq:St}
\end{equation}



\BiSection{参考文献管理}{Reference Management}
参考文献的具体内容就是 reference 文件夹下的 .bib,参考文献的元数据 (名称、作者、出处等) 以一定的格式保存在这些纯文本文件中。.bib 文件也可以理解为参考文献的“数据库”,正文中所有引用的参考文件条目都会从这些文件中“析出”。控制参考文献条目“表现形式”(格式) 的是.bst 文件。.bst 文件定义了参考文献风格,使用不同的参考文献风格能将同一个参考文献条目输出成不同的格式。当然,一个文档只能使用一个参考文献风格。按照学校要求,本模板使用的是国标 GBT7714 风格的参考文献。
\BiSubsection{Latex中参考文献的引用与插入方法}{Citation and Insertion of References in Latex}
bib数据库中的参考文献条目可以手动编写,也可以在 Google 的学术搜索中找到。各大数据库也支持将参考文献信息导出为.bib,省时省力。以 Google 学术搜索为例:在“学术搜索设置”中,将“文献管理软件”设为“显示导入 BibTeX”的连接,保存退出。然后学术搜索找到文献后会有“导出到BibTeX”连接,点击后会打开新的标签页。插入文献详细的介绍点击链接\url{https://b23.tv/G1oMWUA},从22:48时开始是讲述怎么插入参考文献的。\par
其中需要注意的是由于中英文参考文献处理起来有差异,所以需要在参考文献中标注是否是中文文献。确切地说,BibTeX并不具有区分中英文参考文献的能力。.bib是“参考文献的内容”,而控制参考文献格式的是.bst文件,本模板附带的是GBT7714-2005NLang.bst。GBT7714-2005NLang.bst中规定:.bib中的条目,如果条目的“Language”域非空,就被认为是中文参考文献,采取一些针对中文的处理方式。\par
例如在如下一段话中插入参考文献,用\textbackslash cite\{文献标识\}在文中引用文献:
关于主题法的起源众说不一。国内有人认为“主题法检索体系的形式和发展开始于1856年英国克雷斯塔多罗(Crestadoro)的《图书馆编制目录技术》一书”,“国外最早采用主题法来组织目录索引的是杜威十进分类法的相关主题索引……” \cite{Jiang2005Size}。也有人认出为“美国的贝加逊·富兰克林出借图书馆第一个使用了主题法”\cite{Takahashi1996Structure,Xia2002Analysis,Jiang1989}。

\BiSection{定理与定义}{Definition and Proof}
定理与定义的写入很简单,方法如下:
\begin{lstlisting}[caption={定理写入代码}]
\begin{thm}
设函数$y=f(x)$在区间(a,b)上可导,它对应曲线是向上凹(或向下凹)的充分必要条件是:导数 $y=f^{'}(x)$在区间(a,b)上是单调增(或单调减)。
\end{thm}
\end{lstlisting}
\begin{thm}
	设函数$y=f(x)$在区间(a,b)上可导,它对应曲线是向上凹(或向下凹)的充分必要条件是:导数 $y=f^{'}(x)$在区间(a,b)上是单调增(或单调减)。
\end{thm}
\begin{lstlisting}[caption={定义写入代码}]
\begin{defn}[函数极值]
设函数$f(x)$在区间(a,b)内有定义,$x_0$是(a,b)内一点。\par
若存在着$x_0$点的一个邻域,对于这个邻域内任何点$x$($x_0$点除外),$f(x)<f(x_{0})$均成立,则说$f(x_{0})$ 是函数 $f(x)$的一个极大值;若存在着$x_0$点的一个邻域,对于这个邻域内任何点$x$($x_0$点除外),$f(x)>f(x_{0})$均成立,则说$f(x_{0})$ 是函数$f(x)$ 的一个极小值. 函数的极大值与极小值统称为函数的极值。
\end{defn}
\end{lstlisting}
\begin{defn}[函数极值]
	设函数$f(x)$在区间(a,b)内有定义,$x_0$是(a,b)内一点。\par
	若存在着$x_0$点的一个邻域,对于这个邻域内任何点$x$($x_0$点除外),$f(x)<f(x_{0})$均成立,则说$f(x_{0})$ 是函数 $f(x)$的一个极大值;若存在着$x_0$点的一个邻域,对于这个邻域内任何点$x$($x_0$点除外),$f(x)>f(x_{0})$均成立,则说$f(x_{0})$ 是函数$f(x)$ 的一个极小值. 函数的极大值与极小值统称为函数的极值。
\end{defn}

\BiSection{算法代码的插入}{Code Insertion Method}
论文中算法代码的插入示例如下:
\begin{lstlisting}[caption={算法代码的插入示例}]
\begin{algorithm}[h]  
   \caption{Pseudocode of Simulated Annealing Algorithm} % 名称 
   \begin{algorithmic}[1]   
     \Require      
       $x_0$: initial individual or state;     
       $T_0$: a high enough initial temperature;      
       $T_{min}$: the lowest limit of temperature;    
     \Ensure       
       optimal state or approximate optimal state;       
       \State set $x_0 = x_{best}$, compute initial energy function $E(x_0)$;       
       \While {$T > T_{min}$}        
        \For{$i = 1$; $i<n$; $i++$ }      
       \State perturb current state $x_i$ for a new state $x_{new}$ and compute energy function $E(x_{new})$;      
       \State compute $\Delta$ = $E(x_{new}-E(x_{(i)}))$;      
       \If {$\Delta$$E<0$} \State $x_{best} = x_{new}$      
       \Else \State the probability $P = exp(-dE/T_{(i)})$;      
       \If {$rand(0,1) < P$ }\State $x_{best} = x_{new}$      
       \Else \State $x_{best} = x_{best}$      
       \EndIf     
      \EndIf     
      \EndFor      
       \State $T = T * $ $ \alpha$, where $\alpha$ is decay factor;
     \EndWhile  
   \end{algorithmic}
\end{algorithm}
\end{lstlisting}

\begin{algorithm}[h]  
	\caption{Pseudocode of Simulated Annealing Algorithm} % 名称 
	\begin{algorithmic}[1]   
	  \Require      
		$x_0$: initial individual or state;     
		$T_0$: a high enough initial temperature;      
		$T_{min}$: the lowest limit of temperature;    
      \Ensure       
		optimal state or approximate optimal state;       
		\State set $x_0 = x_{best}$, compute initial energy function $E(x_0)$;       
		\While {$T > T_{min}$}        
		 \For{$i = 1$; $i<n$; $i++$ }      
		\State perturb current state $x_i$ for a new state $x_{new}$ and compute energy function $E(x_{new})$;      
		\State compute $\Delta$ = $E(x_{new}-E(x_{(i)}))$;      
		\If {$\Delta$$E<0$} \State $x_{best} = x_{new}$      
		\Else \State the probability $P = exp(-dE/T_{(i)})$;      
		\If {$rand(0,1) < P$ }\State $x_{best} = x_{new}$      
		\Else \State $x_{best} = x_{best}$      
		\EndIf     
       \EndIf     
	   \EndFor      
		\State $T = T * $ $ \alpha$, where $\alpha$ is decay factor;
	  \EndWhile  
	\end{algorithmic}
\end{algorithm}
\BiSection{规范表达注意事项}{The Standard Expression}
\BiSubsection{名词术语}{Terminology}
应使用全国自然科学名词审定委员会审定的自然科学名词术语;应按有关的标准或规定使用工程技术名词术语;应使用公认共知的尚无标准或规定的名词术语。作者自拟的名词术语,在文中第一次出现时,须加注说明。表示同一概念或概念组合的名词术语,全文中要前后一致。外国人名可使用原文,不必译出。一般的机关、团体、学校、研究机构和企业等的名称,在论文中第一次出现时必须写全称。
\BiSubsection{数字}{Figures}
数字的使用必须符合新的国家标准GB/T15835-1995《出版物上数字用法的规定》。
\BiSubsection{外文字母}{Foreign Letters}
文中出现的易混淆的字母、符号以及上下标等,必须打印清楚或缮写工整。要严格区分外文字母的文种、大小写、正斜体和黑白体等,必要时用铅笔注明,尤其注意上下标字母的大小写、正斜体。\par
(1) 斜体\par
斜体外文字母用于表示量的符号,主要用于下列场合:\par
1) 变量符号、变动附标及函数。\par
2) 用字母表示的数及代表点、线、面、体和图形的字母。\par
3) 特征数符号,如Re(雷诺数)、Fo(傅里叶数)、Al(阿尔芬数)等。\par
4) 在特定场合中视为常数的参数。\par
5) 矢量、矩阵用黑体斜体。\par
(2) 正体\par
正体外文字母用于表示名称及与其有关的代号,主要用于下列场合:\par
1) 有定义的已知函数(例如sin, exp, ln等)。\par
2) 其值不变的数学常数(例如e=2.718 281 8…)及已定义的算子。\par
3) 法定计量单位、词头和量纲符号。\par
4) 数学符号。\par
5) 化学元素符号。\par
6) 机具、仪器、设备和产品等的型号、代号及材料牌号。\par
7) 硬度符号。\par
8) 不表示量的外文缩写字。\par
9) 表示序号的拉丁字母。\par
10) 量符号中为区别其它量而加的具有特定含义的非量符号下角标。
\BiSubsection{量和单位}{Quantities and Units}
文中涉及的量和单位一律采用新的国家标准GB3100~3102-93《量和单位》。\par
(1) 必须符合国家标准规定,不得使用已废弃的单位,如高斯(G和Gg) 、亩、克分子浓度(M)、当量能度(N)等。\par
(2) 量和单位不用中文名称,而用法定符号表示。
\BiSubsection{标量与向量}{The scalar and Vector}
标量要采用正体,而向量要采用黑体。
\BiSection{本章小结}{The Chapter’s Conclusion}
